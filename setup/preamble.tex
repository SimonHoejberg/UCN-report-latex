%\documentclass[11pt,twoside,a4paper,openright]{report}
\documentclass[11pt, onesided, a4paper]{report}
%%%%%%%%%%%%%%%%%%%%%%%%%%%%%%%%%%%%%%%%%%%%%%%%
% Language, Encoding and Fonts
% http://en.wikibooks.org/wiki/LaTeX/Internationalization
%%%%%%%%%%%%%%%%%%%%%%%%%%%%%%%%%%%%%%%%%%%%%%%%
% Select encoding of your inputs. Depends on
% your operating system and its default input
% encoding. Typically, you should use
%   Linux  : utf8 (most modern Linux distributions)
%            latin1 
%   Windows: ansinew
%            latin1 (works in most cases)
%   Mac    : applemac
% Notice that you can manually change the input
% encoding of your files by selecting "save as"
% an select the desired input encoding. 

\usepackage[utf8]{inputenc}
% Make latex understand and use the typographic
% rules of the language used in the document.
\usepackage[danish]{babel}
\usepackage{tikz}
\usepackage{aeguill}
\usepackage{textcomp}

% Use the palatino font
\usepackage[sc]{mathpazo}
\usepackage{colortbl}
\linespread{1.05}         % Palatino needs more leading (space between lines)

% Choose the font encoding 
\usepackage[T1]{fontenc}
%%%%%%%%%%%%%%%%%%%%%%%%%%%%%%%%%%%%%%%%%%%%%%%%
% Graphics and Tables
% http://en.wikibooks.org/wiki/LaTeX/Importing_Graphics
% http://en.wikibooks.org/wiki/LaTeX/Tables
% http://en.wikibooks.org/wiki/LaTeX/Colors
%%%%%%%%%%%%%%%%%%%%%%%%%%%%%%%%%%%%%%%%%%%%%%%%
% load a colour package
\usepackage{xcolor}
\definecolor{ucnblue}{RGB}{0, 66, 80}% dark blue
% The standard graphics inclusion package
\usepackage{graphicx}
% Set up how figure and table captions are displayed
\usepackage{caption}
\captionsetup{%
  font=footnotesize,% set font size to footnotesize
  labelfont=bf % bold label (e.g., Figure 3.2) font
}
\usepackage{float}
% Make the standard latex tables look so much better
\usepackage{array,booktabs}
% Enable the use of frames around, e.g., theorems
% The framed package is used in the example environment
\usepackage{framed}

%%%%%%%%%%%%%%%%%%%%%%%%%%%%%%%%%%%%%%%%%%%%%%%%
% Mathematics
% http://en.wikibooks.org/wiki/LaTeX/Mathematics
%%%%%%%%%%%%%%%%%%%%%%%%%%%%%%%%%%%%%%%%%%%%%%%%
% Defines new environments such as equation,
% align and split 
\usepackage{amsmath}
% Adds new math symbols
\usepackage{amssymb}
% Use theorems in your document
% The ntheorem package is also used for the example environment
% When using thmmarks, amsmath must be an option as well. Otherwise \eqref doesn't work anymore.
\usepackage[framed,amsmath,thmmarks]{ntheorem}

%%%%%%%%%%%%%%%%%%%%%%%%%%%%%%%%%%%%%%%%%%%%%%%%
% Page Layout
% http://en.wikibooks.org/wiki/LaTeX/Page_Layout
%%%%%%%%%%%%%%%%%%%%%%%%%%%%%%%%%%%%%%%%%%%%%%%%
% Change margins, papersize, etc of the document
\usepackage[
  inner=25mm,% left margin on an odd page
  outer=25mm,% right margin on an odd page
  bottom=30mm,
  top=25mm
  % footskip=20mm
  ]{geometry}
% Modify how \chapter, \section, etc. look
% The titlesec package is very configureable
\usepackage{titlesec}
\titleformat{\chapter}[display]{\normalfont\huge\bfseries}{\chaptertitlename\ \thechapter}{20pt}{\Huge}
\titleformat*{\section}{\normalfont\Large\bfseries}
\titleformat*{\subsection}{\normalfont\large\bfseries}
\titleformat*{\subsubsection}{\normalfont\normalsize\bfseries}
%\titleformat*{\paragraph}{\normalfont\normalsize\bfseries}
%\titleformat*{\subparagraph}{\normalfont\normalsize\bfseries}

% Clear empty pages between chapters
\let\origdoublepage\cleardoublepage
\newcommand{\clearemptydoublepage}{%
  \clearpage
  {\pagestyle{empty}\origdoublepage}%
}
\let\cleardoublepage\clearemptydoublepage

% Change the headers and footers
\usepackage{fancyhdr}
\pagestyle{fancy}
\fancyhf{} %delete everything
\renewcommand{\headrulewidth}{0pt} %remove the horizontal line in the header
\renewcommand{\footrulewidth}{0pt}
% \fancyhead[RE]{\small\nouppercase\leftmark} %even page - chapter title
% \fancyhead[LO]{\small\nouppercase\rightmark} %uneven page - section title
\fancyhead[R]{\thepage\ af \pageref{LastPage}}
% \fancyhead[LE,RO]{\thepage} %page number on all pages
% Do not stretch the content of a page. Instead,
% insert white space at the bottom of the page
% \raggedbottom
% Enable arithmetics with length. Useful when
% typesetting the layout.

\fancypagestyle{plain}{%
  \fancyhf{}%
  \fancyhead[R]{\thepage\ af \pageref{LastPage}}%
  \renewcommand{\headrulewidth}{0pt}% Line at the header invisible
  \renewcommand{\footrulewidth}{0pt}% Line at the footer visible
}

\usepackage{calc}
\usepackage{tabu}
%%%%%%%%%%%%%%%%%%%%%%%%%%%%%%%%%%%%%%%%%%%%%%%%
% Bibliography
% http://en.wikibooks.org/wiki/LaTeX/Bibliography_Management
%%%%%%%%%%%%%%%%%%%%%%%%%%%%%%%%%%%%%%%%%%%%%%%%
% \usepackage[backend=bibtex,
%   bibencoding=utf8
%   ]{biblatex}
% %\addbibresource{bib/lib}
% \bibliography{bib/mybib}
\usepackage[square,numbers]{natbib}
% Appearance of the bibliography
\bibliographystyle{unsrt}
%Sets the bibliography to allign right
\usepackage{etoolbox}
\apptocmd{\thebibliography}{\raggedright}{}{} 
%%%%%%%%%%%%%%%%%%%%%%%%%%%%%%%%%%%%%%%%%%%%%%%%
% Misc
%%%%%%%%%%%%%%%%%%%%%%%%%%%%%%%%%%%%%%%%%%%%%%%%
% Add bibliography and index to the table of
% contents
\usepackage[nottoc]{tocbibind}
% Add the command \pageref{LastPage} which refers to the
% page number of the last page
\usepackage{lastpage}
% Add todo notes in the margin of the document
\setlength{\marginparwidth}{2cm}
\usepackage[
%  disable, %turn off todonotes
  colorinlistoftodos, %enable a coloured square in the list of todos
  textwidth=\marginparwidth, %set the width of the todonotes
  textsize=scriptsize, %size of the text in the todonotes
  ]{todonotes}

%%%%%%%%%%%%%%%%%%%%%%%%%%%%%%%%%%%%%%%%%%%%%%%%
% Hyperlinks
% http://en.wikibooks.org/wiki/LaTeX/Hyperlinks
%%%%%%%%%%%%%%%%%%%%%%%%%%%%%%%%%%%%%%%%%%%%%%%%
% Enable hyperlinks and insert info into the pdf
% file. Hypperref should be loaded as one of the 
% last packages
\usepackage{hyperref}
\hypersetup{%
	pdfpagelabels=true,%
	plainpages=false,%
	pdfauthor={Bjarke Mac Sporring,
	Hans Christian Skaksen,
	Rasmus Molbech og
 	Simon Højberg },%
	pdftitle={Semester Projekt},%
	pdfsubject={Semester Projekt},%
	bookmarksnumbered=true,%
	colorlinks=false,%
	citecolor=black,%
	filecolor=black,%
	linkcolor=black,% you should probably change this to black before printing
    hidelinks=true,
	urlcolor=black,%
	pdfstartview=FitH%
}
\usepackage[scaled]{beramono}
\usepackage{listings}
\usepackage{rotating}
\usepackage{color}
\definecolor{mygreen}{rgb}{0,0.6,0}
\definecolor{mygray}{rgb}{0.5,0.5,0.5}
\definecolor{bluekeywords}{rgb}{0.13,0.13,1}
\definecolor{redstring}{rgb}{0.6,0,0}
\definecolor{javaKeywords}{HTML}{7F0055}
\definecolor{diff}{HTML}{c5c5c5}
\definecolor{bluekeywords}{rgb}{0.13,0.13,1}
\definecolor{greencomments}{rgb}{0,0.5,0}
\definecolor{redstrings}{rgb}{0.9,0,0}

%\definecolor{white}{rgb}{1,1,1}

\lstdefinestyle{nc}%
{
  %morecomment  = [l]{//},
  %morecomment  = [l][\nullfont]{//},
  %morecomment  = [is]{/*}{*/}
}

\lstdefinestyle{CSharp}{
  language=[Sharp]C,
  frame=ltrb,
  framesep=5pt,
  captionpos=b,
  numbers=left,
  numberstyle=\tiny,
  showspaces=false,
  showtabs=false,
  breaklines=true,
  showstringspaces=false,
  breakatwhitespace=true,
  escapeinside={(*@}{@*)},
  commentstyle=\color{greencomments},
  morekeywords={partial, var, value, get, set, async, await},
  basicstyle=\linespread{0.9}\ttfamily\small,
  keywordstyle=\color{bluekeywords},
  stringstyle=\color{redstrings},
  literate= {Å}{{\AA}}1 {å}{{\aa}}1 {Æ}{{\AE}}1 {æ}{{\ae}}1 
}

% \renewcommand\lstlistingname{Code fragment}

\lstdefinestyle{sql}{
  language=SQL,
  frame=ltrb,
  framesep=5pt,
  basicstyle=\linespread{0.9}\ttfamily\small,
  keywordstyle=\ttfamily\color{bluekeywords},
  identifierstyle=\ttfamily, 
  stringstyle=\ttfamily,
  breaklines=true,
  numbers=left,
  numberstyle=\tiny,
  commentstyle=\ttfamily\color{greencomments}\small,
  basicstyle=\small,
  morekeywords= {LANGUAGE, BEGIN, RETURN, RETURNS, FUNCTION, IF, REPLACE},
  captionpos=b,
  morekeywords={TEXT, BIGINT}
  }

\definecolor{lightgray}{rgb}{.9,.9,.9}
\definecolor{darkgray}{rgb}{.4,.4,.4}
\definecolor{purple}{rgb}{0.65, 0.12, 0.82}
\lstdefinelanguage{JavaScript}{
  keywords={break, case, catch, continue, debugger, default, delete, do, else, false, finally, for, function, if, in, instanceof, new, null, return, switch, this, throw, true, try, typeof, var, void, while, with},
  morecomment=[l]{//},
  morecomment=[s]{/*}{*/},
  morestring=[b]',
  morestring=[b]",
  ndkeywords={class, export, boolean, throw, implements, import, this},
  basicstyle=\linespread{0.9}\ttfamily\small,
  keywordstyle=\color{blue}\bfseries,
  ndkeywordstyle=\color{darkgray}\bfseries,
  identifierstyle=\color{black},
  commentstyle=\color{purple}\ttfamily,
  stringstyle=\color{red}\ttfamily,
  sensitive=true
}

\lstdefinestyle{JavaScript}{
	language=JavaScript,
	frame=ltrb,
	framesep=5pt,
	basicstyle=\linespread{0.9}\ttfamily\small,
	%backgroundcolor=\color{lightgray},
	extendedchars=true,
	showstringspaces=false,
	showspaces=false,
	numbers=left,
	numberstyle=\footnotesize,
	numbersep=9pt,
	tabsize=4,
	breaklines=true,
	showtabs=false,
	captionpos=b
}

%\usepackage[automake, toc]{glossaries}
%\makeglossaries

\usepackage{pgfplots}
\pgfplotsset{compat=1.14}
\usepackage{float}
\usepackage{wrapfig}

