% see, e.g., http://en.wikibooks.org/wiki/LaTeX/Customizing_LaTeX#New_commands
% for more information on how to create macros

%%%%%%%%%%%%%%%%%%%%%%%%%%%%%%%%%%%%%%%%%%%%%%%%
% Macros for the titlepage
%%%%%%%%%%%%%%%%%%%%%%%%%%%%%%%%%%%%%%%%%%%%%%%%
%Creates the ucn titlepage
\newcommand{\ucntitlepage}[3]{%
  {
    %set up various length
    \ifx\titlepageleftcolumnwidth\undefined
      \newlength{\titlepageleftcolumnwidth}
      \newlength{\titlepagerightcolumnwidth}
    \fi
    \setlength{\titlepageleftcolumnwidth}{0.8\textwidth-\tabcolsep}
    \setlength{\titlepagerightcolumnwidth}{\textwidth-2\tabcolsep-\titlepageleftcolumnwidth}
    %create title page
    \thispagestyle{empty}
    \noindent%
    \begin{tabular}{@{}ll@{}}
      \parbox{\titlepageleftcolumnwidth}{ 
        \iflanguage{danish}{%
          \includegraphics[width=\titlepageleftcolumnwidth]{figures/Logo/UCN-Logo}
        }{%
          \includegraphics[width=\titlepageleftcolumnwidth]{figures/Logo/UCN-Logo}
        }
      } &
      \parbox{\titlepagerightcolumnwidth}{\raggedleft\sf\small
        #2
      }\bigskip\\
       #1 &
      \parbox[t]{\titlepagerightcolumnwidth}{%
      %\textbf{Abstract:}\bigskip\par
        \parbox{\titlepagerightcolumnwidth-2\fboxsep-2\fboxrule}{%
          #3
        }
      }\\
    \end{tabular}
    \vfill
    \clearpage
  }
}

%Create english project info
\newcommand{\englishprojectinfo}[8]{%
  \parbox[t]{\titlepageleftcolumnwidth}{
    \textbf{Title:}\\ #1\bigskip\par
    \textbf{Theme:}\\ #2\bigskip\par
    \textbf{Project Period:}\\ #3\bigskip\par
    \textbf{Project Group:}\\ #4\bigskip\par
    \textbf{Participant(s):}\\ #5\bigskip\par
    \textbf{Supervisor(s):}\\ #6\bigskip\par
    \textbf{Copies:} #7\bigskip\par
    \textbf{Page Numbers:} \pageref{LastPage}\bigskip\par
    \textbf{Date of Completion:}\\ #8
  }
}

%Create danish project info
\newcommand{\danishprojectinfo}[8]{%
  \parbox[t]{\titlepageleftcolumnwidth}{
    \textbf{Titel:}\\ #1\bigskip\par
    \textbf{Projektperiode:}\\ #2\bigskip\par
    \textbf{Projektgruppe:}\\ #3\bigskip\par
    \textbf{Deltagere:}\\ #4\bigskip\par
    \textbf{Vejleder:}\\ #5\bigskip\par
    \textbf{Normalsider:}\\ #6\bigskip\par
    \textbf{Repository:}\\ #7\bigskip\par
    \textbf{Afleveringsdato:}\\ #8
  }
}

%%%%%%%%%%%%%%%%%%%%%%%%%%%%%%%%%%%%%%%%%%%%%%%%
% An example environment
%%%%%%%%%%%%%%%%%%%%%%%%%%%%%%%%%%%%%%%%%%%%%%%%
\theoremheaderfont{\normalfont\bfseries}
\theorembodyfont{\normalfont}
\theoremstyle{break}
\def\theoremframecommand{{\color{gray!50}\vrule width 5pt \hspace{5pt}}}
\newshadedtheorem{exa}{Example}[chapter]
\newenvironment{example}[1]{%
		\begin{exa}[#1]
}{%
		\end{exa}
}

\newcommand{\tabitem}{~~\llap{\textbullet}~~}

\newcommand{\fig}[3][]{
\begin{figure}[H]
\begin{center}
  \ifthenelse{\equal{#1}{}}
  {\includegraphics[keepaspectratio=true,width=\textwidth]{figures/#2}}
  {\includegraphics[keepaspectratio=true,scale=#1]{figures/#2}}
  
  \caption{#3}
  \label{#2}
\end{center}
\end{figure}
}

% Used in frontpage. Please don't use elsewhere.
\newcommand{\figx}[2][0.5]{
\begin{figure}[H]
\begin{center}
  \includegraphics[keepaspectratio=true,scale=#1]{figures/#2}
\end{center}
\end{figure}
}

\newcommand{\figr}[2]{
\begin{wrapfigure}{r}{0.4\textwidth}
\begin{center}
  \includegraphics[keepaspectratio=true,width=0.4\textwidth]{figures/#1}
  \caption{#2}
  \label{#1}
\end{center}
\end{wrapfigure}
}

\newcommand{\inputS}[1]{
  \input{sections/#1.tex}
}

\newcommand{\tofix}[1]{
  \todo[linecolor=red, backgroundcolor=red]{#1}
}

\newcommand{\worry}[1]{
  \todo[linecolor=yellow, backgroundcolor=yellow]{#1}
}

\newcommand{\toread}[1]{
  \todo[linecolor=blue, backgroundcolor=blue]{#1}
}
